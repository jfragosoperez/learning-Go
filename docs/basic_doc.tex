\documentclass[twoside,a4paper,english]{report}

\usepackage[margin=2cm]{geometry}

\usepackage[utf8]{inputenc}
\usepackage[T1]{fontenc}
\usepackage{xcolor}
\usepackage{amsmath}
%\usepackage{graphicx}
\usepackage{listings}
\usepackage{titlesec}

\usepackage{afterpage}

\titleformat{\chapter}[hang]{\bfseries\huge}{\thechapter.}{2pc}{}
\titlelabel{\thetitle.\quad}   % For consistency in all headings

\newcommand\blankpage{%
    \null
    \thispagestyle{empty}%
    \addtocounter{page}{-1}%
    \newpage}

%\DeclareGraphicsExtensions{.pdf,.png,.jpg}

\title{An Introduction To Programming In Go}
\author{Fragoso Pérez, Jonathan}

\date{\today}

\definecolor{my_blue}{HTML}{0066FF}
\definecolor{my_green}{HTML}{33CC33}
\definecolor{my_gray}{HTML}{669966}

\lstnewenvironment{go}
  {\lstset{language=Go,breaklines=true, frame=lines, commentstyle=\color{my_gray}}}
  {}

\begin{document}
\renewcommand{\abstractname}{Abstract}

\maketitle

\begin{abstract}
Describing the basics of the Go programming language. This document is more like a resume, a quick reference to everyone that has programmed for some time and wants to introduce fastly to this language, without reading many links to the docs or a book.
Some of the information/examples is directly copied from the book \cite{doxsey2012introduction}.
\end{abstract}

%\tableofcontents

\newpage
\chapter{Introduction}
\emph{Why Go? } 

\chapter{Basic Types}
\section{Numbers}
\begin{itemize}
\item Integer
\begin{itemize}
\item uint8 $\rightarrow$ same as byte
\item uint32	
\item uint64
\item int8 
\item int16
\item int32 $\rightarrow$ same as rune
\item int64
\end{itemize}
Machine dependent $\rightarrow$ their size is dependent on the type of architecture of the machine -> uint, int, uintptr \\\\
Go allows to increment/decrement by a unit using the operator ++/--. The language also enables to increment/decrement using the operator +=/-= .
\item Float
\begin{itemize}
\item float32 $\rightarrow$ single precision
\item float64 $\rightarrow$ double precision
\end{itemize}
\item Complex
\begin{itemize}
\item complex64
\item complex128
\end{itemize}
\end{itemize}
\section{Strings}
Some operations:
\begin{itemize}
\item Length $\rightarrow$ len("Hello world")
\item Char. accessing -> "Hello World"[1] $\rightarrow$ returns 101 instead of e as a character is represented as a byte.  
\item Concatenation $\rightarrow$ "Hello " + " world!"	
\end{itemize}
\section{Booleans}
1 bit integer representing true or false.\\\\
Operations:
\begin{itemize}
\item \&\&
\item ||
\item !
\end{itemize}
\chapter{Other Types}
\section{Arrays}
\begin{go}
var integerArray [10]int

x := [5]float64{ 2, 5, 3, 1}
\end{go}
\section{Slices}
Are like arrays, but their size is allowed to change.\\
% {\color{my_blue}Example:}\\
\begin{go}
//slice associated with an underlying
//float64 array of length 5
x := make([]float64, 5)

//slice associated with an underlying
//float64 array of length 5,
//where 10 is the capacity of the underlying
//array which the slice points to
x := make([]float64, 5, 10)

arr := []float64{1,2,3,4,5}
x := arr[0:4] // this will assign to x values [1,2,3,4] because the high 
              // index is not included
\end{go}
Built-in functions:\\
\begin{itemize}
\item Append $\rightarrow$ creates a new slice by taking an existing one.
\begin{go}
slice1 := []int{1,2,3}
slice2 := append(slice1, 4, 5)  

// RESULT:
// slice 1 value is [1,2,3]
// slice 2 value is [1,2,3,4,5]
\end{go} 
\item Copy
\begin{go}
slice1 := []int{1,2,3}
slice2 := make([]int, 2)
copy(slice2, slice1)

// RESULT:
// slice 2 now will have values [1,2] because slice2 has room for only two elements
\end{go}
\end{itemize}
\newpage
\bibliographystyle{plain}
\bibliography{bibliography}

\end{document}


